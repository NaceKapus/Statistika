\documentclass[12pt, letterpaper, twoside]{article}
\usepackage[utf8]{inputenc}
\usepackage[slovene]{babel}
\usepackage{amsmath}

\title{SEMINARSKA NALOGA IZ STATISTIKE  \\
	\Large UL FMF, Matematika --- univerzitetni študij}
\author{Nace Kapus}
\date{2017/18}
 
\begin{document}
 
\begin{titlepage}
\maketitle
\end{titlepage}

\begin{enumerate}
	\item 
	\begin{enumerate}
		\item Verjetnost, da posamezen anketiranec odgovori z DA, je enaka:
		\begin{align*}
		P(\text{DA})&=P(\text{DA na delikatno vprašanje})P(\text{delikatno vprašanje}) \\
		&+P(\text{DA na nevtralno vprašanje})P(\text{nevtralno vprašanje}) \\
		&=pq+(1-p)r:=s.
		\end{align*}
		Da je ta verjetnost enaka za vse anketirance, sledi iz naših predpostavk.
		\item Označimo z $I_i$ indikator dogodka, da i-ti anketiranec odgovori DA. Potem lahko $S$ zapišemo kot vsoto indikatorjev na sledeč način:
		$$S = \frac{1}{n}\sum_{i=1}^{n}I_i.$$
		Računamo:
		\begin{align*}
		E(S)&=E\left(\frac{1}{n}\sum_{i=1}^{n}I_i\right) 
		=\frac{1}{n}\sum_{i=1}^{n}E(I_i) \\
		&=\frac{1}{n}\sum_{i=1}^{n}s 
		=\frac{1}{n} \cdot ns \\
		&=s
		\end{align*}
		Torej je $S$ res nepristranska cenilka za $s$.
		\item \begin{align*}
			\text{var}(S)&=\text{var}\left(\frac{1}{n}\sum_{i=1}^{n}I_i\right)=\frac{1}{n^2}\text{ var}\left(\sum_{i=1}^{n}I_i\right) \\
			&=\frac{1}{n^2}\left(\sum_{i=1}^{n}\text{var}(I_i)+\sum_{\underset{i\neq j}{i,j=1}}^{n}\text{cov}(I_i,I_j) \right)
		\end{align*}
		Ker so $I_i$ indikatorji oziroma Bernoullijeve slučajne spremenljivke, zanje velja: $E(I_i)=P(I_i)=s$
		in var$(I_i)=s(1-s)$. Izračunajmo še cov$(I_i,I_j)$:
		$$
		\text{cov}(I_i,I_j)=E(I_i I_j)-E(I_i)E(I_j)
		$$
		Naj bo N velikost populacije. $E(I_i I_j) = P(I_i = 1,I_j =1)$, kar se lahko zgodi na več načinov:
		\begin{itemize}
			\item obema je dodeljeno delikatno vprašanje in oba imata lastnost A
			\item obema je dodeljeno nevtralno vprašanje in oba imata lastnost B
			\item nekdo dobi delikatno vprašanje in ima lastnost A, drugi dobi nevtralno vprašanje in ima lastnost B
		\end{itemize}
		\begin{align*}
		E(I_i I_j) &= P(I_i = 1,I_j =1) \\
		&= q \cdot \frac{qN-1}{N-1}\cdot p^2 \\ &+(1-q)\cdot\frac{(1-q)N-1}{N-1}\cdot(1-p)^2 \\ &+2q\cdot\frac{(1-q)N}{N-1}\cdot p(1-p) \\
		&=\frac{N}{N-1}s^2-\frac{qp^2+(1-q)(1-p)^2}{N-1}
		\end{align*}
		Če povzamemo:
		\begin{align*}
			\text{var}(S) &=\frac{1}{n^2}\left(\sum_{i=1}^{n}\text{var}(I_i)+\sum_{\underset{i\neq j}{i,j=1}}^{n}\text{cov}(I_i,I_j) \right) \\
			&=\frac{1}{n^2}\left(\sum_{i=1}^{n}s(1-s)+\sum_{\underset{i\neq j}{i,j=1}}^{n}\left(\frac{N}{N-1}s^2-\frac{qp^2+(1-q)(1-p)^2}{N-1}-s^2\right)\right) \\
			&=\frac{1}{n}\left(s(1-s)+(n-1)\left(\frac{N}{N-1}s^2-\frac{qp^2+(1-q)(1-p)^2}{N-1}-s^2\right)\right)
		\end{align*}
		
		\item Kadar gre $N \longrightarrow \infty$, gre cov$(I_i,I_j)$ proti 0. V tem primeru velja, da je
		$$\text{var}(S)\sim\frac{s(1-s)}{n}.$$
		
		\item Nepristranska cenilka za $q$ obstaja v primeru, ko je $p$ različen od 0 (vsaj kdo dobi delikatno vprašanje). Iz enakosti $$s=pq+(1-p)r $$
		lahko izrazimo cenilko za $q$, in sicer:
		$$\overset{\wedge}{q}=\frac{S-(1-p)r}{p}.$$
		\item Iz prejšnje točke sledi:
		$$\text{var}(\overset{\wedge}{q})=\text{var}\left(\frac{S}{p}\right)\sim\frac{s(1-s)}{np^2}.$$
		
	\end{enumerate}
	\item Povprečni dohodek izračunamo iz dohodkov družin, katere smo izbrali s slučajnim vzorčenjem. Če bi vzorčili ponovno, bi (najverjetneje) izbrali druge družine ter posledično dobili drugačen rezultat. Torej je to res slučajna spremenljivka.
	\item
	\item
	\item 
	\begin{enumerate}
		\item \begin{align*}
		Y_i & = \beta_0 + \beta_1 x_i + \varepsilon_i ~~/ : \rho_i  \\
		\frac{Y_i}{\rho_i} &= \beta_0 \frac{1}{\rho_i}+\beta_1 \frac{x_i}{\rho_i}+\frac{\varepsilon_i}{\rho_i} \\
		Z_i &= \beta_0 c_i + \beta_1 d_i + e_i
		\end{align*}
		Da nov model zadošča predpostavkam standardne linearne regresije, mora zanj veljati: $E(e_i)=0$, var$(e_i)=\sigma^2$, ter cov$(e_i,e_j)=0$ za $i\neq j.$ Računamo:
		\begin{align*}
		&E(e_i) = E\left(\frac{\varepsilon_i}{\rho_i}\right) =\frac{1}{\rho_i}\cdot 0 = 0 \\
		&\text{var}(e_i)=\text{var}\left(\frac{\epsilon_i}{\rho_i}\right)=\frac{1}{\rho_i^2}\text{ var}(\varepsilon_i)=\frac{\rho_i^2 \sigma^2}{\rho_i^2}=\sigma^2 \\
		&\text{cov}(e_i,e_j)=\text{cov}\left(\frac{\varepsilon_i}{\rho_i},\frac{\varepsilon_j}{\rho_j}\right)=\frac{1}{\rho_i \rho_j}\text{ cov}(\varepsilon_i,\varepsilon_j)=0
		\end{align*}
		Matrika modela je enaka 
		 \[ X =
		\begin{bmatrix}
		\frac{1}{\rho_1}       & \frac{x_1}{\rho_1}  \\
		\frac{1}{\rho_2}       & \frac{x_2}{\rho_2}  \\
		\vdots				   & \vdots \\
		\frac{1}{\rho_n}       & \frac{x_n}{\rho_n}  
		\end{bmatrix}
		\]
		\item V novem modelu minimiziramo količino
		$$\sum_{i=1}^{n}(z_i-\beta_o c_i-\beta_1 d_i)^2.$$
		Računamo:
		$$\sum_{i=1}^{n}(z_i-\beta_0 c_i-\beta_1 d_i)^2 =\sum_{i=1}^{n}\left(\frac{Y_i}{\rho_i}-\beta_0 \frac{1}{\rho_i}-\beta_1 \frac{x_i}{\rho_i}\right)^2=\sum_{i=1}^{n}\frac{(Y_i-\beta_0-\beta_1 x_i)^2}{\rho_i^2}$$
		Komentar: funkcija, ki jo želimo minimizirati v novem modelu je zelo podobna funkciji izhodiščnega modela, pojavijo se le faktorji $\frac{1}{\rho_i^2}$. To pomeni, da imajo v funkciji, ki jo minimiziramo, večji pomen opažanja, ki imajo manjše variance in obratno.
		\item Ker imamo standarden linearni regresijski model vemo: 
		 \[
		\begin{bmatrix}
		\overset{\wedge}{\beta_0}        \\
		\overset{\wedge}{\beta_1}      
		\end{bmatrix}
		= (X^T X)^{-1}X^Tz,\]
		pri čemer je 
		 \[ X =
		 \begin{bmatrix}
		 \frac{1}{\rho_1}       & \frac{x_1}{\rho_1}  \\
		 \frac{1}{\rho_2}       & \frac{x_2}{\rho_2}  \\
		 \vdots				   & \vdots \\
		 \frac{1}{\rho_n}       & \frac{x_n}{\rho_n}  
		 \end{bmatrix}
		 in ~ z =
		  \begin{bmatrix}
		  \frac{y_1}{\rho_1}        \\
		  \frac{y_2}{\rho_2}         \\
		  \vdots				   \\
		  \frac{y_n}{\rho_n}        
		  \end{bmatrix}
		  .\]
		  Izračunamo:
		  \[ X^TX =
		  \begin{bmatrix}
		  \sum_{i=1}^{n}\frac{1}{\rho_i^2}       & \sum_{i=1}^{n}\frac{x_i}{\rho_i^2}  \\
		  \sum_{i=1}^{n}\frac{x_i}{\rho_i^2}       & \sum_{i=1}^{n}\frac{x_i^2}{\rho_i^2} 
		  \end{bmatrix}
		  , ~ X^Tz =
		  \begin{bmatrix}
		  \sum_{i=1}^{n}\frac{y_i}{\rho_i^2}        \\
		  \sum_{i=1}^{n}\frac{x_iy_i}{\rho_i^2}       
		  \end{bmatrix}
		  .\]
		  Od tod sledi:
		 $$ \overset{\wedge}{\beta_0}= \frac{ \sum_{i=1}^{n}\frac{x_i^2}{\rho_i^2}\sum_{i=1}^{n}\frac{y_i}{\rho_i^2}-\sum_{i=1}^{n}\frac{x_i}{\rho_i^2}\sum_{i=1}^{n}\frac{x_iy_i}{\rho_i^2}}{\sum_{i=1}^{n}\frac{1}{\rho_i^2}\sum_{i=1}^{n}\frac{x_i^2}{\rho_i^2}-\sum_{i=1}^{n}\frac{x_i}{\rho_i^2}\sum_{i=1}^{n}\frac{x_i}{\rho_i^2}}  $$\\
		 $$ \overset{\wedge}{\beta_1}= \frac{-\sum_{i=1}^{n}\frac{x_i}{\rho_i^2}\sum_{i=1}^{n}\frac{y_i}{\rho_i^2}+\sum_{i=1}^{n}\frac{1}{\rho_i^2}\sum_{i=1}^{n}\frac{x_iy_i}{\rho_i^2}}{\sum_{i=1}^{n}\frac{1}{\rho_i^2}\sum_{i=1}^{n}\frac{x_i^2}{\rho_i^2}-\sum_{i=1}^{n}\frac{x_i}{\rho_i^2}\sum_{i=1}^{n}\frac{x_i}{\rho_i^2}}. $$
		\item Po formuli iz predavanj:
		$$\text{var}\left(\overset{\wedge}{\beta_0}\right) = \sigma^2 \left[\left(X^TX\right)^{-1}\right]_{11} =\sigma^2\cdot \frac{\sum_{i=1}^{1}\frac{x_i^2}{\rho_i^2}}{\sum_{i=1}^{n}\frac{1}{\rho_i^2}\sum_{i=1}^{n}\frac{x_i^2}{\rho_i^2}-\sum_{i=1}^{n}\frac{x_i}{\rho_i^2}\sum_{i=1}^{n}\frac{x_i}{\rho_i^2}}$$ \\
		$${var}\left(\overset{\wedge}{\beta_1}\right) =\sigma^2 \left[\left(X^TX\right)^{-1}\right]_{22}=\sigma^2\cdot \frac{\sum_{i=1}^{1}\frac{1}{\rho_i^2}}{\sum_{i=1}^{n}\frac{1}{\rho_i^2}\sum_{i=1}^{n}\frac{x_i^2}{\rho_i^2}-\sum_{i=1}^{n}\frac{x_i}{\rho_i^2}\sum_{i=1}^{n}\frac{x_i}{\rho_i^2}}$$
		
		
	\end{enumerate}
\end{enumerate}
 
\end{document}
